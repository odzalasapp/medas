% Options for packages loaded elsewhere
\PassOptionsToPackage{unicode}{hyperref}
\PassOptionsToPackage{hyphens}{url}
%
\documentclass[
  french,
]{article}
\usepackage{lmodern}
\usepackage{amssymb,amsmath}
\usepackage{ifxetex,ifluatex}
\ifnum 0\ifxetex 1\fi\ifluatex 1\fi=0 % if pdftex
  \usepackage[T1]{fontenc}
  \usepackage[utf8]{inputenc}
  \usepackage{textcomp} % provide euro and other symbols
\else % if luatex or xetex
  \usepackage{unicode-math}
  \defaultfontfeatures{Scale=MatchLowercase}
  \defaultfontfeatures[\rmfamily]{Ligatures=TeX,Scale=1}
\fi
% Use upquote if available, for straight quotes in verbatim environments
\IfFileExists{upquote.sty}{\usepackage{upquote}}{}
\IfFileExists{microtype.sty}{% use microtype if available
  \usepackage[]{microtype}
  \UseMicrotypeSet[protrusion]{basicmath} % disable protrusion for tt fonts
}{}
\makeatletter
\@ifundefined{KOMAClassName}{% if non-KOMA class
  \IfFileExists{parskip.sty}{%
    \usepackage{parskip}
  }{% else
    \setlength{\parindent}{0pt}
    \setlength{\parskip}{6pt plus 2pt minus 1pt}}
}{% if KOMA class
  \KOMAoptions{parskip=half}}
\makeatother
\usepackage{xcolor}
\IfFileExists{xurl.sty}{\usepackage{xurl}}{} % add URL line breaks if available
\IfFileExists{bookmark.sty}{\usepackage{bookmark}}{\usepackage{hyperref}}
\hypersetup{
  pdftitle={Projet Approches spatiales et temporelles des données},
  pdfauthor={Prince ODZALAMBAE \& Elhadji SOW},
  hidelinks,
  pdfcreator={LaTeX via pandoc}}
\urlstyle{same} % disable monospaced font for URLs
\usepackage{color}
\usepackage{fancyvrb}
\newcommand{\VerbBar}{|}
\newcommand{\VERB}{\Verb[commandchars=\\\{\}]}
\DefineVerbatimEnvironment{Highlighting}{Verbatim}{commandchars=\\\{\}}
% Add ',fontsize=\small' for more characters per line
\usepackage{framed}
\definecolor{shadecolor}{RGB}{248,248,248}
\newenvironment{Shaded}{\begin{snugshade}}{\end{snugshade}}
\newcommand{\AlertTok}[1]{\textcolor[rgb]{0.94,0.16,0.16}{#1}}
\newcommand{\AnnotationTok}[1]{\textcolor[rgb]{0.56,0.35,0.01}{\textbf{\textit{#1}}}}
\newcommand{\AttributeTok}[1]{\textcolor[rgb]{0.77,0.63,0.00}{#1}}
\newcommand{\BaseNTok}[1]{\textcolor[rgb]{0.00,0.00,0.81}{#1}}
\newcommand{\BuiltInTok}[1]{#1}
\newcommand{\CharTok}[1]{\textcolor[rgb]{0.31,0.60,0.02}{#1}}
\newcommand{\CommentTok}[1]{\textcolor[rgb]{0.56,0.35,0.01}{\textit{#1}}}
\newcommand{\CommentVarTok}[1]{\textcolor[rgb]{0.56,0.35,0.01}{\textbf{\textit{#1}}}}
\newcommand{\ConstantTok}[1]{\textcolor[rgb]{0.00,0.00,0.00}{#1}}
\newcommand{\ControlFlowTok}[1]{\textcolor[rgb]{0.13,0.29,0.53}{\textbf{#1}}}
\newcommand{\DataTypeTok}[1]{\textcolor[rgb]{0.13,0.29,0.53}{#1}}
\newcommand{\DecValTok}[1]{\textcolor[rgb]{0.00,0.00,0.81}{#1}}
\newcommand{\DocumentationTok}[1]{\textcolor[rgb]{0.56,0.35,0.01}{\textbf{\textit{#1}}}}
\newcommand{\ErrorTok}[1]{\textcolor[rgb]{0.64,0.00,0.00}{\textbf{#1}}}
\newcommand{\ExtensionTok}[1]{#1}
\newcommand{\FloatTok}[1]{\textcolor[rgb]{0.00,0.00,0.81}{#1}}
\newcommand{\FunctionTok}[1]{\textcolor[rgb]{0.00,0.00,0.00}{#1}}
\newcommand{\ImportTok}[1]{#1}
\newcommand{\InformationTok}[1]{\textcolor[rgb]{0.56,0.35,0.01}{\textbf{\textit{#1}}}}
\newcommand{\KeywordTok}[1]{\textcolor[rgb]{0.13,0.29,0.53}{\textbf{#1}}}
\newcommand{\NormalTok}[1]{#1}
\newcommand{\OperatorTok}[1]{\textcolor[rgb]{0.81,0.36,0.00}{\textbf{#1}}}
\newcommand{\OtherTok}[1]{\textcolor[rgb]{0.56,0.35,0.01}{#1}}
\newcommand{\PreprocessorTok}[1]{\textcolor[rgb]{0.56,0.35,0.01}{\textit{#1}}}
\newcommand{\RegionMarkerTok}[1]{#1}
\newcommand{\SpecialCharTok}[1]{\textcolor[rgb]{0.00,0.00,0.00}{#1}}
\newcommand{\SpecialStringTok}[1]{\textcolor[rgb]{0.31,0.60,0.02}{#1}}
\newcommand{\StringTok}[1]{\textcolor[rgb]{0.31,0.60,0.02}{#1}}
\newcommand{\VariableTok}[1]{\textcolor[rgb]{0.00,0.00,0.00}{#1}}
\newcommand{\VerbatimStringTok}[1]{\textcolor[rgb]{0.31,0.60,0.02}{#1}}
\newcommand{\WarningTok}[1]{\textcolor[rgb]{0.56,0.35,0.01}{\textbf{\textit{#1}}}}
\usepackage{longtable,booktabs}
% Correct order of tables after \paragraph or \subparagraph
\usepackage{etoolbox}
\makeatletter
\patchcmd\longtable{\par}{\if@noskipsec\mbox{}\fi\par}{}{}
\makeatother
% Allow footnotes in longtable head/foot
\IfFileExists{footnotehyper.sty}{\usepackage{footnotehyper}}{\usepackage{footnote}}
\makesavenoteenv{longtable}
\usepackage{graphicx,grffile}
\makeatletter
\def\maxwidth{\ifdim\Gin@nat@width>\linewidth\linewidth\else\Gin@nat@width\fi}
\def\maxheight{\ifdim\Gin@nat@height>\textheight\textheight\else\Gin@nat@height\fi}
\makeatother
% Scale images if necessary, so that they will not overflow the page
% margins by default, and it is still possible to overwrite the defaults
% using explicit options in \includegraphics[width, height, ...]{}
\setkeys{Gin}{width=\maxwidth,height=\maxheight,keepaspectratio}
% Set default figure placement to htbp
\makeatletter
\def\fps@figure{htbp}
\makeatother
\setlength{\emergencystretch}{3em} % prevent overfull lines
\providecommand{\tightlist}{%
  \setlength{\itemsep}{0pt}\setlength{\parskip}{0pt}}
\setcounter{secnumdepth}{5}

%encoding
%--------------------------------------
\usepackage[utf8]{inputenc}
\usepackage[T1]{fontenc}
%--------------------------------------
 
%French-specific commands
%--------------------------------------
\usepackage[autolanguage]{numprint}
%--------------------------------------
 
%Hyphenation rules
%--------------------------------------
\usepackage{hyphenat}

%Style page
%--------------------------------------
\usepackage{booktabs}
\usepackage{amsmath}
\usepackage{longtable}
\usepackage{array}
\usepackage{multirow}
\usepackage{wrapfig}
\usepackage{float}
\usepackage{colortbl}
\usepackage{pdflscape}
\usepackage{tabu}
\usepackage{threeparttable}
\usepackage{threeparttablex}
\usepackage[normalem]{ulem}
\usepackage{makecell}
\usepackage{xcolor}
\usepackage{sectsty}
\usepackage{graphicx}
\graphicspath{ {images/} }
\usepackage{caption}
\usepackage{subcaption}

\usepackage{hyperref}
\hypersetup{
    colorlinks = true,
	linkcolor=black,
	urlcolor = {red},
    linkbordercolor = {white}
}

\usepackage[left=25mm,top=30mm,right=25mm,bottom=30mm,bindingoffset=6mm]{geometry}

\usepackage{fancyhdr}
\pagestyle{fancy}
\fancyhead{}
\fancyfoot{}
\renewcommand{\headrulewidth}{0.4pt}
\renewcommand{\footrulewidth}{0.4pt}
 
\fancyhf{}
\lhead{USID06 - Approches spatiales et temporelles des données}
\rhead{M1 MEDAS}
\lfoot{CNAM Nantes}
\rfoot{Page \thepage}

\usepackage{framed,color}
\definecolor{shadecolor}{RGB}{248,248,248}

%Supprime le titre du Rmarkdown
%--------------------------------------
\AtBeginDocument{\let\maketitle\relax}
\usepackage{booktabs}
\usepackage{longtable}
\usepackage{array}
\usepackage{multirow}
\usepackage{wrapfig}
\usepackage{float}
\usepackage{colortbl}
\usepackage{pdflscape}
\usepackage{tabu}
\usepackage{threeparttable}
\usepackage{threeparttablex}
\usepackage[normalem]{ulem}
\usepackage{makecell}
\usepackage{xcolor}
\ifxetex
  % Load polyglossia as late as possible: uses bidi with RTL langages (e.g. Hebrew, Arabic)
  \usepackage{polyglossia}
  \setmainlanguage[]{french}
\else
  \usepackage[shorthands=off,main=french]{babel}
\fi

\title{Projet Approches spatiales et temporelles des données}
\author{Prince ODZALAMBAE \& Elhadji SOW}
\date{jeudi, 20 août 2020}

\begin{document}
\maketitle

\thispagestyle{empty}
\begin{center}
	\vspace*{1cm}
	
	\Huge
	\textsf{Projet Approches spatiales et temporelles des données}
	
	\vspace{0.5cm}
	
	\textsf{Prince ODZALAMBAE}
	\textsf{Elhadji SOW}
	
	\vspace{0.5cm}
	\LARGE
	Jeudi, 20 Août 2020
	
	\vfill
	
	Approches spatiales et temporelles des données\\
	USID06
	
	\vspace{1cm}
	
	\includegraphics[width=0.4\textwidth]{cnam}
	
	\Large
	MASTER 1 MEDAS\\
	2019/2020
\end{center}

\setlength{\abovedisplayskip}{-5pt}
\setlength{\abovedisplayshortskip}{-5pt}

\newpage

{
\setcounter{tocdepth}{2}
\tableofcontents
}
\newpage

\hypertarget{pruxe9sentation}{%
\section*{Présentation}\label{pruxe9sentation}}
\addcontentsline{toc}{section}{Présentation}

\begin{itemize}
\tightlist
\item
  \textbf{Problématique}
\end{itemize}

La problématique de notre sujet est de savoir si la construction de logement sociaux sur les communes du département ces dernières années contribue ou pas à la mixité sociale.

\begin{itemize}
\tightlist
\item
  \textbf{Objectif}
\end{itemize}

L'objectif de ce projet est de répondre à la problématique en fournissant une vue générale montrant l'implantation géographique du logement social et une une analyse approfondie sur le département de la Loire-Atlantique.

\begin{itemize}
\tightlist
\item
  \textbf{Réalisation}
\end{itemize}

Ce projet a été réalisé uniquement sur \textbf{R Studio}. Nous avons créé un projet \textbf{R Markdown} avec l'extension \textbf{bookdown} et une application \textbf{Shiny}, affin de pouvoir intégrer sous forme de slide des traitements effectués avec R, tout en favorisant la reproductibilité des calculs et la génération du rapport dans le format souhaiter \textbf{HTML, PDF, EBUP}.

\newpage

\hypertarget{introduction}{%
\section{Introduction}\label{introduction}}

Depuis une dizaine d'années, en France comme au Québec, la fixation de taux minimum de logements sociaux (par commune en France et par opération immobilière au Québec) est devenue l'un des principaux instruments des pouvoirs publics pour favoriser le développement du parc social, dans un contexte de baisse des financements gouvernementaux dédiés à la construction (Kamoun, 2005; Driant, 2010; Laberge et Montmarquette, 2010) mais aussi d'injonction croissante à la mixité sociale (Palomares, 2008; Kirszbaum, 2008)

En France, après de longs et tumultueux débats, la loi relative à la solidarité et au renouvellement urbain (sru) du 13 décembre 2000 a rendu obligatoire la présence d'un minimum de 20\% de logements sociaux dans toutes les communes d'au moins 3 500 habitants (1 500 en Île-de-France). La récente loi du 24 mars 2014 pour l'accès au logement et un urbanisme rénové (Alur) a élevé ce seuil à 25\%.

Ce document comportera les parties suivantes :

Pour cela, nous vous présentera un tableau de board composée de graphiques.
L'objectif est de fournir une vue générale (composée de graphiques de données) montrant l'implantation géographique du logement social sur le territoire.

Description du corpus, traitement des données (décrire les étapes du processus de traitement et intégrer le code correspondant), les choix des typologies visuelles et des outils employés sont à expliciter.

Activation des packages
Récupération des données
Datapréparation
On tronque filtre sur les données de ventes de maisons en filtrant les données à 98\% pour lisser les moyennes
Calcul de l'évolution des prix et du nombre de ventes
A l'epci
A la commune

Intégration des données aux fonds de carte
Datavisualisation
Carte à l'EPCI de la région
zoom à la commune

Présentation des visualisations de données réalisées (légendes apportant des précisions), tableaux de bords.

Interprétation, confrontation de vos objectifs de départ avec les résultats obtenus.

Analyse critique et retour sur le travail réalisé. Qu'avez-vous découvert de plus ? Que pourriez-vous améliorer ?

\newpage

\hypertarget{part-objectifs}{%
\part{Objectifs}\label{part-objectifs}}

\hypertarget{objectifs}{%
\section{Objectifs}\label{objectifs}}

Here is a review of existing methods.

\hypertarget{part-muxe9thodes}{%
\part{Méthodes}\label{part-muxe9thodes}}

\hypertarget{muxe9thodes}{%
\section{Méthodes}\label{muxe9thodes}}

We describe our methods in this chapter.

When you subset a data frame, it does not necessarily return
a data frame. For example, if you subset two columns, you get
a data frame, but when you try to subset one column, you get
a vector:

\begin{Shaded}
\begin{Highlighting}[]
\NormalTok{mtcars[}\DecValTok{1}\OperatorTok{:}\DecValTok{5}\NormalTok{, }\StringTok{"mpg"}\NormalTok{]}
\end{Highlighting}
\end{Shaded}

\begin{verbatim}
[1] 21.0 21.0 22.8 21.4 18.7
\end{verbatim}

To make sure that we always get a data frame, we have to use
the argument \texttt{drop\ =\ FALSE}. Now we use the chunk option
\texttt{class.source\ =\ "bg-success"}.

\begin{Shaded}
\begin{Highlighting}[]
\NormalTok{mtcars[}\DecValTok{1}\OperatorTok{:}\DecValTok{5}\NormalTok{, }\StringTok{"mpg"}\NormalTok{, drop =}\StringTok{ }\OtherTok{FALSE}\NormalTok{]}
\end{Highlighting}
\end{Shaded}

\begin{tabular}{l|r}
\hline
  & mpg\\
\hline
Mazda RX4 & 21.0\\
\hline
Mazda RX4 Wag & 21.0\\
\hline
Datsun 710 & 22.8\\
\hline
Hornet 4 Drive & 21.4\\
\hline
Hornet Sportabout & 18.7\\
\hline
\end{tabular}

Then we assign a class \texttt{watch-out} to the code chunk via the
chunk option \texttt{class.source}.

\begin{Shaded}
\begin{Highlighting}[]
\NormalTok{mtcars[}\DecValTok{1}\OperatorTok{:}\DecValTok{5}\NormalTok{, }\StringTok{"mpg"}\NormalTok{]}
\end{Highlighting}
\end{Shaded}

\begin{verbatim}
[1] 21.0 21.0 22.8 21.4 18.7
\end{verbatim}

We have defined some CSS rules to limit the height of
code blocks. Now we can test if these rules work on code
blocks and text output:

\begin{Shaded}
\begin{Highlighting}[]
\CommentTok{# pretend that we have a lot of code in this chunk}
\ControlFlowTok{if}\NormalTok{ (}\DecValTok{1} \OperatorTok{+}\StringTok{ }\DecValTok{1} \OperatorTok{==}\StringTok{ }\DecValTok{2}\NormalTok{) \{}
  \CommentTok{# of course that is true}
  \KeywordTok{print}\NormalTok{(mtcars)}
  \CommentTok{# we just printed a lengthy data set}
\NormalTok{\}}
\end{Highlighting}
\end{Shaded}

\begin{verbatim}
                     mpg cyl  disp  hp drat    wt  qsec vs am gear carb
Mazda RX4           21.0   6 160.0 110 3.90 2.620 16.46  0  1    4    4
Mazda RX4 Wag       21.0   6 160.0 110 3.90 2.875 17.02  0  1    4    4
Datsun 710          22.8   4 108.0  93 3.85 2.320 18.61  1  1    4    1
Hornet 4 Drive      21.4   6 258.0 110 3.08 3.215 19.44  1  0    3    1
Hornet Sportabout   18.7   8 360.0 175 3.15 3.440 17.02  0  0    3    2
Valiant             18.1   6 225.0 105 2.76 3.460 20.22  1  0    3    1
Duster 360          14.3   8 360.0 245 3.21 3.570 15.84  0  0    3    4
Merc 240D           24.4   4 146.7  62 3.69 3.190 20.00  1  0    4    2
Merc 230            22.8   4 140.8  95 3.92 3.150 22.90  1  0    4    2
Merc 280            19.2   6 167.6 123 3.92 3.440 18.30  1  0    4    4
Merc 280C           17.8   6 167.6 123 3.92 3.440 18.90  1  0    4    4
Merc 450SE          16.4   8 275.8 180 3.07 4.070 17.40  0  0    3    3
Merc 450SL          17.3   8 275.8 180 3.07 3.730 17.60  0  0    3    3
Merc 450SLC         15.2   8 275.8 180 3.07 3.780 18.00  0  0    3    3
Cadillac Fleetwood  10.4   8 472.0 205 2.93 5.250 17.98  0  0    3    4
Lincoln Continental 10.4   8 460.0 215 3.00 5.424 17.82  0  0    3    4
Chrysler Imperial   14.7   8 440.0 230 3.23 5.345 17.42  0  0    3    4
Fiat 128            32.4   4  78.7  66 4.08 2.200 19.47  1  1    4    1
Honda Civic         30.4   4  75.7  52 4.93 1.615 18.52  1  1    4    2
Toyota Corolla      33.9   4  71.1  65 4.22 1.835 19.90  1  1    4    1
Toyota Corona       21.5   4 120.1  97 3.70 2.465 20.01  1  0    3    1
Dodge Challenger    15.5   8 318.0 150 2.76 3.520 16.87  0  0    3    2
AMC Javelin         15.2   8 304.0 150 3.15 3.435 17.30  0  0    3    2
Camaro Z28          13.3   8 350.0 245 3.73 3.840 15.41  0  0    3    4
Pontiac Firebird    19.2   8 400.0 175 3.08 3.845 17.05  0  0    3    2
Fiat X1-9           27.3   4  79.0  66 4.08 1.935 18.90  1  1    4    1
Porsche 914-2       26.0   4 120.3  91 4.43 2.140 16.70  0  1    5    2
Lotus Europa        30.4   4  95.1 113 3.77 1.513 16.90  1  1    5    2
Ford Pantera L      15.8   8 351.0 264 4.22 3.170 14.50  0  1    5    4
Ferrari Dino        19.7   6 145.0 175 3.62 2.770 15.50  0  1    5    6
Maserati Bora       15.0   8 301.0 335 3.54 3.570 14.60  0  1    5    8
Volvo 142E          21.4   4 121.0 109 4.11 2.780 18.60  1  1    4    2
\end{verbatim}

Next we add rules for a new class \texttt{scroll-100} to limit
the height to 100px, and add the class to the output of
a code chunk via the chunk option \texttt{class.output}:

\begin{Shaded}
\begin{Highlighting}[]
\KeywordTok{print}\NormalTok{(mtcars)}
\end{Highlighting}
\end{Shaded}

\begin{verbatim}
                     mpg cyl  disp  hp drat    wt  qsec vs am gear carb
Mazda RX4           21.0   6 160.0 110 3.90 2.620 16.46  0  1    4    4
Mazda RX4 Wag       21.0   6 160.0 110 3.90 2.875 17.02  0  1    4    4
Datsun 710          22.8   4 108.0  93 3.85 2.320 18.61  1  1    4    1
Hornet 4 Drive      21.4   6 258.0 110 3.08 3.215 19.44  1  0    3    1
Hornet Sportabout   18.7   8 360.0 175 3.15 3.440 17.02  0  0    3    2
Valiant             18.1   6 225.0 105 2.76 3.460 20.22  1  0    3    1
Duster 360          14.3   8 360.0 245 3.21 3.570 15.84  0  0    3    4
Merc 240D           24.4   4 146.7  62 3.69 3.190 20.00  1  0    4    2
Merc 230            22.8   4 140.8  95 3.92 3.150 22.90  1  0    4    2
Merc 280            19.2   6 167.6 123 3.92 3.440 18.30  1  0    4    4
Merc 280C           17.8   6 167.6 123 3.92 3.440 18.90  1  0    4    4
Merc 450SE          16.4   8 275.8 180 3.07 4.070 17.40  0  0    3    3
Merc 450SL          17.3   8 275.8 180 3.07 3.730 17.60  0  0    3    3
Merc 450SLC         15.2   8 275.8 180 3.07 3.780 18.00  0  0    3    3
Cadillac Fleetwood  10.4   8 472.0 205 2.93 5.250 17.98  0  0    3    4
Lincoln Continental 10.4   8 460.0 215 3.00 5.424 17.82  0  0    3    4
Chrysler Imperial   14.7   8 440.0 230 3.23 5.345 17.42  0  0    3    4
Fiat 128            32.4   4  78.7  66 4.08 2.200 19.47  1  1    4    1
Honda Civic         30.4   4  75.7  52 4.93 1.615 18.52  1  1    4    2
Toyota Corolla      33.9   4  71.1  65 4.22 1.835 19.90  1  1    4    1
Toyota Corona       21.5   4 120.1  97 3.70 2.465 20.01  1  0    3    1
Dodge Challenger    15.5   8 318.0 150 2.76 3.520 16.87  0  0    3    2
AMC Javelin         15.2   8 304.0 150 3.15 3.435 17.30  0  0    3    2
Camaro Z28          13.3   8 350.0 245 3.73 3.840 15.41  0  0    3    4
Pontiac Firebird    19.2   8 400.0 175 3.08 3.845 17.05  0  0    3    2
Fiat X1-9           27.3   4  79.0  66 4.08 1.935 18.90  1  1    4    1
Porsche 914-2       26.0   4 120.3  91 4.43 2.140 16.70  0  1    5    2
Lotus Europa        30.4   4  95.1 113 3.77 1.513 16.90  1  1    5    2
Ford Pantera L      15.8   8 351.0 264 4.22 3.170 14.50  0  1    5    4
Ferrari Dino        19.7   6 145.0 175 3.62 2.770 15.50  0  1    5    6
Maserati Bora       15.0   8 301.0 335 3.54 3.570 14.60  0  1    5    8
Volvo 142E          21.4   4 121.0 109 4.11 2.780 18.60  1  1    4    2
\end{verbatim}

\hypertarget{part-ruxe9sultats}{%
\part{Résultats}\label{part-ruxe9sultats}}

\hypertarget{ruxe9sultats}{%
\section{Résultats}\label{ruxe9sultats}}

Some \emph{significant} applications are demonstrated in this chapter.

L'article 55 de la loi SRU de 2002 vise à promouvoir la mixité sociale en imposant un seuil de logements sociaux (20 \%, passé en 2013 à 25 \%) aux communes des grandes agglomérations. Une étude dans le département des Yvelines montre que les villes déficitaires en logement social s'accommodent finalement bien de ces objectifs mais que la mixité sociale n'est toujours pas au rendez-vous.

\begin{Shaded}
\begin{Highlighting}[]
\CommentTok{# Carreaux Insee ----------------------}
\CommentTok{# Source : https://www.insee.fr/fr/statistiques/4176290?sommaire=4176305}
\NormalTok{df <-}\StringTok{ }\KeywordTok{FiltrerZone}\NormalTok{(}\DataTypeTok{depcom =} \StringTok{'44109'}\NormalTok{)}

\KeywordTok{save}\NormalTok{(df,}\DataTypeTok{file =} \StringTok{'data/carreaux_insee_nantes.RData'}\NormalTok{)}
\CommentTok{# Source : https://www.data.gouv.fr/fr/datasets/repertoire-des-logements-locatifs-des-bailleurs-sociaux/#_}
\NormalTok{rpls_}\DecValTok{2019}\NormalTok{_nantes <-}\StringTok{ }\KeywordTok{fread}\NormalTok{(}\StringTok{'extdata/rpls_geoloc_2019/RPLS2019_detail_reg52.csv'}\NormalTok{) }\OperatorTok\StringTok{ }
\StringTok{  }\KeywordTok{filter}\NormalTok{(DEPCOM }\OperatorTok{==}\StringTok{ '44109'}\NormalTok{)}

\KeywordTok{save}\NormalTok{(rpls_}\DecValTok{2019}\NormalTok{_nantes,}\DataTypeTok{file =} \StringTok{'data/rpls_2019_nantes.RData'}\NormalTok{)}
\end{Highlighting}
\end{Shaded}

This R Markdown document is made interactive using Shiny. Unlike the more traditional workflow of creating static reports, you can now create documents that allow your readers to change the assumptions underlying your analysis and see the results immediately.

To learn more, see \href{http://rmarkdown.rstudio.com/authoring_shiny.html}{Interactive Documents}.

\begin{Shaded}
\begin{Highlighting}[]
\KeywordTok{library}\NormalTok{(leaflet)}
\KeywordTok{leaflet}\NormalTok{() }\OperatorTok
\StringTok{  }\KeywordTok{setView}\NormalTok{(}\FloatTok{174.764}\NormalTok{, }\FloatTok{-36.877}\NormalTok{, }\DataTypeTok{zoom =} \DecValTok{16}\NormalTok{) }\OperatorTok\StringTok{ }
\StringTok{  }\KeywordTok{addTiles}\NormalTok{() }\OperatorTok
\StringTok{  }\KeywordTok{addMarkers}\NormalTok{(}\FloatTok{174.764}\NormalTok{, }\FloatTok{-36.877}\NormalTok{, }\DataTypeTok{popup =} \StringTok{"Maungawhau"}\NormalTok{) }
\end{Highlighting}
\end{Shaded}

\includegraphics{figures/unnamed-chunk-8-1.pdf}

\begin{Shaded}
\begin{Highlighting}[]
\KeywordTok{library}\NormalTok{(dygraphs)}
\NormalTok{lungDeaths <-}\StringTok{ }\KeywordTok{cbind}\NormalTok{(mdeaths, fdeaths)}
\KeywordTok{dygraph}\NormalTok{(lungDeaths)}
\end{Highlighting}
\end{Shaded}

\includegraphics{figures/unnamed-chunk-9-1.pdf}

\begin{Shaded}
\begin{Highlighting}[]
\KeywordTok{dygraph}\NormalTok{(lungDeaths) }\OperatorTok\StringTok{ }\KeywordTok{dyRangeSelector}\NormalTok{()}
\end{Highlighting}
\end{Shaded}

\includegraphics{figures/unnamed-chunk-9-2.pdf}

\begin{Shaded}
\begin{Highlighting}[]
\KeywordTok{dygraph}\NormalTok{(lungDeaths) }\OperatorTok
\StringTok{  }\KeywordTok{dySeries}\NormalTok{(}\StringTok{"mdeaths"}\NormalTok{, }\DataTypeTok{label =} \StringTok{"Male"}\NormalTok{) }\OperatorTok
\StringTok{  }\KeywordTok{dySeries}\NormalTok{(}\StringTok{"fdeaths"}\NormalTok{, }\DataTypeTok{label =} \StringTok{"Female"}\NormalTok{) }\OperatorTok
\StringTok{  }\KeywordTok{dyOptions}\NormalTok{(}\DataTypeTok{stackedGraph =} \OtherTok{TRUE}\NormalTok{) }\OperatorTok
\StringTok{  }\KeywordTok{dyRangeSelector}\NormalTok{(}\DataTypeTok{height =} \DecValTok{20}\NormalTok{)}
\end{Highlighting}
\end{Shaded}

\includegraphics{figures/unnamed-chunk-9-3.pdf}

\begin{Shaded}
\begin{Highlighting}[]
\NormalTok{hw <-}\StringTok{ }\KeywordTok{HoltWinters}\NormalTok{(ldeaths)}
\NormalTok{predicted <-}\StringTok{ }\KeywordTok{predict}\NormalTok{(hw, }\DataTypeTok{n.ahead =} \DecValTok{72}\NormalTok{, }\DataTypeTok{prediction.interval =} \OtherTok{TRUE}\NormalTok{)}

\KeywordTok{dygraph}\NormalTok{(predicted, }\DataTypeTok{main =} \StringTok{"Predicted Lung Deaths (UK)"}\NormalTok{) }\OperatorTok
\StringTok{  }\KeywordTok{dyAxis}\NormalTok{(}\StringTok{"x"}\NormalTok{, }\DataTypeTok{drawGrid =} \OtherTok{FALSE}\NormalTok{) }\OperatorTok
\StringTok{  }\KeywordTok{dySeries}\NormalTok{(}\KeywordTok{c}\NormalTok{(}\StringTok{"lwr"}\NormalTok{, }\StringTok{"fit"}\NormalTok{, }\StringTok{"upr"}\NormalTok{), }\DataTypeTok{label =} \StringTok{"Deaths"}\NormalTok{) }\OperatorTok
\StringTok{  }\KeywordTok{dyOptions}\NormalTok{(}\DataTypeTok{colors =}\NormalTok{ RColorBrewer}\OperatorTok{::}\KeywordTok{brewer.pal}\NormalTok{(}\DecValTok{3}\NormalTok{, }\StringTok{"Set1"}\NormalTok{))}
\end{Highlighting}
\end{Shaded}

\includegraphics{figures/unnamed-chunk-9-4.pdf}

\begin{Shaded}
\begin{Highlighting}[]
\CommentTok{#knitr::included_url('miniUI')}
\CommentTok{#knitr::include_app('miniUI/app.R', height = '600px')}
\end{Highlighting}
\end{Shaded}

\hypertarget{inputs-and-outputs}{%
\subsection{Inputs and Outputs}\label{inputs-and-outputs}}

You can embed Shiny inputs and outputs in your document. Outputs are automatically updated whenever inputs change. This demonstrates how a standard R plot can be made interactive by wrapping it in the Shiny \texttt{renderPlot} function. The \texttt{selectInput} and \texttt{sliderInput} functions create the input widgets used to drive the plot.

\hypertarget{embedded-application}{%
\subsection{Embedded Application}\label{embedded-application}}

It's also possible to embed an entire Shiny application within an R Markdown document using the \texttt{shinyAppDir} function. This example embeds a Shiny application located in another directory:

Note the use of the \texttt{height} parameter to determine how much vertical space the embedded application should occupy.

You can also use the \texttt{shinyApp} function to define an application inline rather then in an external directory.

In all of R code chunks above the \texttt{echo\ =\ FALSE} attribute is used. This is to prevent the R code within the chunk from rendering in the document alongside the Shiny components.

\hypertarget{part-analyse}{%
\part{Analyse}\label{part-analyse}}

\hypertarget{analyse}{%
\section{Analyse}\label{analyse}}

We have finished a nice book.

Voici la liste des fichiers externes utilisés dans ce projet.

\hypertarget{mixituxe9-sociale-diversituxe9-sociale-des-concepts-uxe0-la-pratique}{%
\subsection{Mixité sociale, diversité sociale : des concepts à la pratique}\label{mixituxe9-sociale-diversituxe9-sociale-des-concepts-uxe0-la-pratique}}

Les notions de mixité et de diversité sociales
sont défi nies par la loi. Elles fi xent un cadre
pour l'action des organismes Hlm : l'objectif de
diversité de l'habitat résulte de l'article 16 de la
loi d'orientation pour la ville de 1991 et l'objectif
de mixité sociale des villes et des quartiers est
inscrit dans la loi de lutte contre les exclusions
de 1998 et dans le CCH à l'article L411(1).
Bien qu'actées par la loi, la diversité et la mixité
sociales restent des notions diffi ciles à défi nir.
Loin de se réduire à la seule dimension de
l'habitat, elles renvoient à la dimension urbaine
et notamment aux questions de transports, de
services et d'école. Leur promotion exige des
engagements importants pour aller à l'encontre
de la tendance des marchés à la ségrégation
territoriale et au souhait de « l'entre-soi ».
La mixité sociale doit fonctionner comme
une ligne directrice permanente de l'action
locale. Elle renvoie à la question de l'équilibre
permettant la cohésion sociale et le vivre
ensemble, mais aussi à l'égalité des chances.
La Halde défi nit la diversité sociale de manière
large : elle doit être entendue « en termes de
niveaux de revenu et d'éducation, mais aussi en
termes d'origines géographiques, culturelles,
ethniques, d'âge, de compositions familiales,
d'apparence ou de condition physique »(2).
S'il est impossible, et sans doute vain, de
chercher à déterminer le « bon degré » de mixité
ou de diversité pour que ces personnes ou
groupes coexistent harmonieusement, nous
connaissons les effets de l'absence de mixité
sociale et ses conséquences.
D'une façon générale, la réalisation de logements
sociaux nouveaux est une solution pour
permettre de satisfaire une pluralité d'impératifs.
Elle représente un enjeu pour les politiques de
l'habitat : davantage de logements sociaux, bien
répartis sur le territoire des agglomérations pour
y accueillir différentes catégories de ménages,
dont les jeunes et les classes moyennes, ne
pouvant se loger de façon satisfaisante dans
les conditions du marché. Mais également plus
de logements pour accueillir les personnes
défavorisées dans une situation de mixité sociale
des villes et des quartiers.

You can label chapter and section titles using \texttt{\{\#label\}} after them, e.g., we can reference Chapter \ref{sommaire}. If you do not manually label them, there will be automatic labels anyway, e.g., Chapter \ref{methodes}.

\url{https://www.metropolitiques.eu/Logement-social-et-application-de-la-loi-SRU-la-lettre-plutot-que-l-esprit.html}

Figures and tables with captions will be placed in \texttt{figure} and \texttt{table} environments, respectively.

\hypertarget{a-propos-de-ce-document}{%
\section*{A propos de ce document}\label{a-propos-de-ce-document}}
\addcontentsline{toc}{section}{A propos de ce document}

\hypertarget{code-source}{%
\subsection*{Code source}\label{code-source}}
\addcontentsline{toc}{subsection}{Code source}

Ce projet est open source, vous pouvez acceder au code source sur Github à l'adresse suivante : \url{https://github.com/princeodzalasapp/medas/tree/master/projets/approches_spatiales}

Le code source est généré par l'extension \href{https://bookdown.org/yihui/bookdown/}{bookdown} de \href{https://yihui.name/en/}{Yihui Xie}.

\hypertarget{licence}{%
\subsection*{Licence}\label{licence}}
\addcontentsline{toc}{subsection}{Licence}

Ce document est mise à disposition selon les termes de la Licence Creative Commons Attribution - Pas d'Utilisation Commerciale - Partage dans les Mêmes Conditions 4.0 International.

\hypertarget{session-info}{%
\subsection*{Session info}\label{session-info}}
\addcontentsline{toc}{subsection}{Session info}

\begin{verbatim}
 setting  value                       
 version  R version 3.6.1 (2019-07-05)
 os       Windows 10 x64              
 system   x86_64, mingw32             
 ui       RTerm                       
 language (EN)                        
 collate  French_France.1252          
 ctype    French_France.1252          
 tz       Europe/Paris                
 date     2020-08-20                  
\end{verbatim}

\hypertarget{packages}{%
\subsection*{Packages}\label{packages}}
\addcontentsline{toc}{subsection}{Packages}

\begin{table}[H]
\centering
\begin{tabular}{l|l|l}
\hline
package & ondiskversion & source\\
\hline
bookdown & 0.20 & CRAN (R 3.6.3)\\
\hline
cartogram & 0.2.0 & CRAN (R 3.6.3)\\
\hline
cowplot & 1.0.0 & CRAN (R 3.6.3)\\
\hline
data.table & 1.12.6 & CRAN (R 3.6.1)\\
\hline
dplyr & 0.8.3 & CRAN (R 3.6.1)\\
\hline
DT & 0.15 & CRAN (R 3.6.3)\\
\hline
dygraphs & 1.1.1.6 & CRAN (R 3.6.3)\\
\hline
forcats & 0.4.0 & CRAN (R 3.6.1)\\
\hline
ggplot2 & 3.2.1 & CRAN (R 3.6.1)\\
\hline
ggspatial & 1.1.4 & CRAN (R 3.6.3)\\
\hline
glue & 1.3.1 & CRAN (R 3.6.1)\\
\hline
htmlwidgets & 1.5.1 & CRAN (R 3.6.1)\\
\hline
kableExtra & 1.1.0 & CRAN (R 3.6.1)\\
\hline
knitr & 1.26 & CRAN (R 3.6.1)\\
\hline
leaflet & 2.0.3 & CRAN (R 3.6.3)\\
\hline
lwgeom & 0.2.5 & CRAN (R 3.6.3)\\
\hline
mapview & 2.9.0 & CRAN (R 3.6.3)\\
\hline
pacman & 0.5.1 & CRAN (R 3.6.3)\\
\hline
patchwork & 1.0.1 & CRAN (R 3.6.3)\\
\hline
purrr & 0.3.3 & CRAN (R 3.6.1)\\
\hline
readr & 1.3.1 & CRAN (R 3.6.1)\\
\hline
rmapshaper & 0.4.4 & CRAN (R 3.6.3)\\
\hline
scales & 1.1.0 & CRAN (R 3.6.1)\\
\hline
sf & 0.9.5 & CRAN (R 3.6.3)\\
\hline
stringr & 1.4.0 & CRAN (R 3.6.1)\\
\hline
tibble & 2.1.3 & CRAN (R 3.6.1)\\
\hline
tidyr & 1.0.0 & CRAN (R 3.6.1)\\
\hline
tidyverse & 1.3.0 & CRAN (R 3.6.3)\\
\hline
tinytex & 0.17 & CRAN (R 3.6.1)\\
\hline
tmap & 3.1 & CRAN (R 3.6.3)\\
\hline
tmaptools & 3.1 & CRAN (R 3.6.3)\\
\hline
viridis & 0.5.1 & CRAN (R 3.6.1)\\
\hline
viridisLite & 0.3.0 & CRAN (R 3.6.1)\\
\hline
\end{tabular}
\end{table}

\hypertarget{ruxe9fuxe9rences}{%
\subsection*{Références}\label{ruxe9fuxe9rences}}
\addcontentsline{toc}{subsection}{Références}

\begin{enumerate}
\def\labelenumi{\arabic{enumi}.}
\tightlist
\item
  (Allaire et al. \protect\hyperlink{ref-R-rmarkdown}{2020})
\item
  (R Core Team \protect\hyperlink{ref-R-base}{2019})
\item
  (Xie \protect\hyperlink{ref-xie2015}{2015})
\item
  (Xie \protect\hyperlink{ref-R-knitr}{2019})
\item
  (Xie \protect\hyperlink{ref-R-bookdown}{2020})
\end{enumerate}

\hypertarget{refs}{}
\leavevmode\hypertarget{ref-R-rmarkdown}{}%
Allaire, JJ, Yihui Xie, Jonathan McPherson, Javier Luraschi, Kevin Ushey, Aron Atkins, Hadley Wickham, Joe Cheng, Winston Chang, et Richard Iannone. 2020. \emph{rmarkdown: Dynamic Documents for R}. \url{https://CRAN.R-project.org/package=rmarkdown}.

\leavevmode\hypertarget{ref-R-base}{}%
R Core Team. 2019. \emph{R: A Language and Environment for Statistical Computing}. Vienna, Austria: R Foundation for Statistical Computing. \url{https://www.R-project.org/}.

\leavevmode\hypertarget{ref-xie2015}{}%
Xie, Yihui. 2015. \emph{Dynamic Documents with R and knitr}. 2nd éd. Boca Raton, Florida: Chapman; Hall/CRC. \url{http://yihui.org/knitr/}.

\leavevmode\hypertarget{ref-R-knitr}{}%
---------. 2019. \emph{knitr: A General-Purpose Package for Dynamic Report Generation in R}. \url{https://CRAN.R-project.org/package=knitr}.

\leavevmode\hypertarget{ref-R-bookdown}{}%
---------. 2020. \emph{bookdown: Authoring Books and Technical Documents with R Markdown}. \url{https://CRAN.R-project.org/package=bookdown}.

\end{document}
